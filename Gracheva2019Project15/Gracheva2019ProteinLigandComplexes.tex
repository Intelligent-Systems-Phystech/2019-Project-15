\documentclass[12pt,twoside]{article}
\usepackage{jmlda}
%\NOREVIEWERNOTES
\title
    %[Образец оформления статьи для публикации] % Краткое название; не нужно, если полное название влезает в~колонтитул
    {Формулировка и решение задачи оптимизации, сочетающей классификацию и регрессию, 
    для оценки энергии связывания белка и маленьких молекул}
\author
    %[Грачева~А.\,С.] % список авторов для колонтитула; не нужен, если основной список влезает в колонтитул
    {Анастасия Грачева, Мария Кадукова, Сергей Грудинин, В.В. Стрижов} % основной список авторов, выводимый в оглавление
    %[Автор~И.\,О.$^1$, Соавтор~И.\,О.$^2$, Фамилия~И.\,О.$^2$] % список авторов, выводимый в заголовок; не нужен, если он не отличается от основного

\email
    {gracheva.as@phystech.edu}
\organization
    {$^1$ФИВТ МФТИ}
\abstract
    {При разработке лекарства возникает задача поиска маленьких молекул - лигандов, наиболее сильно взаимодействующих с исследуемым белком, а значит являющихся основными кандидатами в лекарства. Один из способов определить действительное положение такого лиганда заключается в том, чтобы генерировать несколько возможных положений и классифицировать их как нативные и не нативные. Но качество предсказания может быть повышено, если использовать экспериментальные данные о свободной энергии связывания молекул и решать одновременно задачи регрессии и классификации. В статье будут рассмотрены эксперименты с алгоритмом, использующим эту идею.
    

%\bigskip
%\textbf{Ключевые слова}: \emph {ключевое слово, ключевое слово, еще ключевые слова}.
}
\titleEng
    {JMLDA paper example: file jmlda-example.tex}
\authorEng
    {Author~F.\,S.$^1$, CoAuthor~F.\,S.$^2$, Name~F.\,S.$^2$}
\organizationEng
    {$^1$Organization; $^2$Organization}
\abstractEng
    {This document is an example of paper prepared with \LaTeXe\
    typesetting system and style file \texttt{jmlda.sty}.

    \bigskip
    %\textbf{Keywords}: \emph{keyword, keyword, more keywords}.
}
\begin{document}
\maketitle
%\linenumbers
\section{Введение}
Предсказание наиболее выгодной ориентации и положения молекул по отношению друг к другу для образования устойчивого комплекса из белка и лиганда, или молекулярный докинг - задача, важная для ускорения процесса разработки новых лекарств.
Есть два метода её решения: pose prediction - среди нескольких сгенерированных положений лиганда в белке определить наиболее близкое к реальному и scoring - предсказать аффинность (свободную энергию связывания) для комплексов различных белков с лигандами. При этом положение с наименьшей энергией связывания будет соответствовать нативной конформации. Первая задача решена в работе [1] с помощью оптимизации скоринговой функции, учитывающей всевозможные комбинации различных пар атомов и расстояния между ними. Раскладывая эту функцию по базису, авторы представляют её как вектор структурных коэффициентов и сводят задачу к модифицированной SVM-классификации.

Наше предположение заключается в том, что с использованием экспериментальных данных об аффинностях можно улучшить качество классификации. В эксперименте, описанном в данной статье, мы проверим эту гипотезу, а также постараемся решать оптимизационную задачу максимально эффективно вычислительно, чтобы использовать как можно больше доступных экспериментальных  данных.

\section{Постановка задачи}
Пусть $\{C_{ij}\}_{i=1}^P$ - комплексы белков и лигандов. При $j=0$ они находятся в нативных позах, при $j = 1 \dots D$ - в ненативных. Задача заключается в том, чтобы найти скоринговую функцию $E$, который удовлетворяет неравенствам: 
\begin{equation}\\
\begin{split}
& E(C_{i0}) < E(C_{ij}) \\
& \forall i\in 1,\dots,P, \\
& \forall j\in 1,\dots,D.
\end{split}
\end{equation}

В модели взаимодействия, описанной в [1], эта функция задаётся скоринговым вектором $\mathbf{w}$. Поэтому неравенство выше может быть преобразовано в систему неравенств:
 
\begin{equation}\label{eq9}
\begin{split}
& \langle\mathbf{x}_{i0}, \mathbf{w}\rangle < \langle\mathbf{x}_{ij}, \mathbf{w}\rangle, \\
& \langle\mathbf{x}_{ij} - \mathbf{x}_{i0}, \mathbf{w}\rangle > 0, \\
& \forall i = 1,\dots, P, \\
& \forall j = 1, \dots, D.
\end{split}
\end{equation}
где $x_{ij}$ - структурные вектора.  

Чтобы гарантировать единственность решения, а также решать задачу в случае линейной неразделимости выборки, приведём её к виду задачи квадратичной оптимизации с мягким зазором:
\begin{equation}\label{eq10}
\begin{aligned}
& \underset{\mathbf{w}, b_i, \xi_{ij}}{\text{minimize:}}
& & \frac{1}{2} \|\mathbf{w}\|^2 + C\sum\limits_{ij}\xi_{ij} \\
& \text{subject to:}
& & y_{ij}[\langle\mathbf{w},\mathbf{x}_{ij}\rangle - b_i]-1+\xi_{ij} \geq 0, \\
&&& \xi_{ij} \geq 0,\\
&&&i\in\{1,\dots,P\}, \\
&&&j\in\{0,\dots,D\},
\end{aligned}
\end{equation}
где $\mathbf{w}$, $b_i$ и переменные невязки $\xi_{ij}$ -- оптимизируемые параметры модели, \\
$y_{i0}=1$ для нативной позы и $y_{ij}=-1, \ j\in\{1,\dots,D\},$ для ненативной,\\
а $C$ -- некоторый коэффициент регуляризации. \\
Таким образом, решив оптимизационную задачу, решаем и задачу класификации.

Кроме того, есть другая потановка этой задачи - задача регрессии, т.е. предсказания значения свободной энергии связывания белка с лигандом:
\begin{equation}\label{eq11}
\begin{aligned}
& \underset{\mathbf{w}}{\text{minimize:}}
& & \sum\limits_{i}[\langle\mathbf{w},\mathbf{x}_{i0}\rangle - s_i]^2 + \alpha\|\mathbf{w}\|^2, \\
&&& i\in\{1,\dots,P\},
\end{aligned}
\end{equation}
где $s_i$ -- экспериментально полученное значение энергии связывания $i$-го нативного соединения, $\alpha$ -- коэффициент регуляризации для ridge-регрессии. 

Объединение этих методов заключается в сложении функций потерь классификации и регрессии:
\begin{equation}\label{eq12}
\begin{aligned}
& \underset{\mathbf{w}, b_i, \xi_{ij}}{\text{minimize:}}
& & \frac{1}{2} \|\mathbf{w}\|^2 + C\sum\limits_{ij}\xi_{ij} + C_{r}\sum\limits_{i} f(\mathbf{x}_{i0},\mathbf{w}, s_i) \\
& \text{subject to:}
& & y_{ij}[\langle\mathbf{w},\mathbf{x}_{ij}\rangle - b_i]-1+\xi_{ij} \geq 0, \\
&&& \xi_{ij} \geq 0, \\
&&&i\in\{1,\dots,P\}, \\
&&&j\in\{0,\dots,D\},
\end{aligned}
\end{equation}
где $f(\mathbf{x}_{i0},\mathbf{w}, s_i)$ -- MSE, $C_r$ -- коэффициент регуляризации для функции потерь регрессии. 

\section{Теоретическая часть}
С помощью замены переменных сведем два квадратичных слагаемых в целевой функции из задачи \eqref{eq12} к одному. 

Функция потерь регрессии MSE для одного комплекса выражается формулой:
\begin{equation}\label{eq15}
f({\mathbf{x}}_{i0}, {\mathbf{w}}, s_i) = (\langle{\mathbf{x}}_{i0}, {\mathbf{w}} \rangle - s_i)^2.
\end{equation}

Тогда для выборки ${\mathbf{X}}=({\mathbf{x}}_{10}, \dots, {\mathbf{x}}_{P0})^{\text{T}}$, состоящей из нативных конфигураций, и целевого вектора  $\mathbf{s}=(s_1,\dots,s_P)^{\text{T}}$ квадратичные слагаемые целевой функции из задачи \eqref{eq12} принимают вид:
\begin{equation*}
\begin{aligned}
& \frac{1}{2} \|{\mathbf{w}}\|^2 + C_{r}\sum\limits_{i} f({\mathbf{x}}_{i0},{\mathbf{w}}, s_i)=\frac{1}{2}{\mathbf{w}}^{\text{T}}{\mathbf{w}}+C_{r}\|{\mathbf{X}}{\mathbf{w}} - \mathbf{s}\|^2 =\\
& = \frac{1}{2}{\mathbf{w}}^{\text{T}}{\mathbf{w}}+ C_{r}{\mathbf{w}}^{\text{T}}{\mathbf{X}}^{\text{T}}{\mathbf{X}}{\mathbf{w}}- 2C_{r}{\mathbf{w}}^{\text{T}}{\mathbf{X}}^{\text{T}}\mathbf{s} + C_{r}\mathbf{s}^{\text{T}}\mathbf{s}=\\
& = {\mathbf{w}}^{\text{T}}\left(\frac{1}{2}\mathbf{I} + C_{r}{\mathbf{X}}^{\text{T}}{\mathbf{X}}\right){\mathbf{w}}- 2C_{r}{\mathbf{w}}^{\text{T}}{\mathbf{X}}^{\text{T}}\mathbf{s}+ C_{r}\mathbf{s}^{\text{T}}\mathbf{s}=\\
& = \left\|\left[\frac{1}{2}\mathbf{I} + C_{r}{\mathbf{X}}^{\text{T}}{\mathbf{X}}\right]^{\frac{1}{2}}{\mathbf{w}}\right\|^2- 2C_{r}\left({\mathbf{w}}^{\text{T}}\left[\frac{1}{2}\mathbf{I} + C_{r}{\mathbf{X}}^{\text{T}}{\mathbf{X}}\right]^{\frac{1}{2}}\right)
\left(\left[\frac{1}{2}\mathbf{I} + C_{r}{\mathbf{X}}^{\text{T}}{\mathbf{X}}\right]^{-\frac{1}{2}}
{\mathbf{X}}^{\text{T}}\mathbf{s}\right)+ C_{r}\mathbf{s}^{\text{T}}\mathbf{s}=\\
& = \left(\left[\frac{1}{2}\mathbf{I} + C_{r}{\mathbf{X}}^{\text{T}}{\mathbf{X}}\right]^{\frac{1}{2}}{\mathbf{w}} - C_r\left[\frac{1}{2}\mathbf{I} + C_{r}{\mathbf{X}}^{\text{T}}{\mathbf{X}}\right]^{-\frac{1}{2}}
{\mathbf{X}}^{\text{T}}\mathbf{s}\right)^2 + C_r\mathbf{s}^{\text{T}}\mathbf{s}- C_r^2\left\|\left[\frac{1}{2}\mathbf{I} + C_{r}{\mathbf{X}}^{\text{T}}{\mathbf{X}}\right]^{-\frac12}
{\mathbf{X}}^{\text{T}}\mathbf{s}\right\|^2.
\end{aligned}
\end{equation*}
Введем замену переменных:
\begin{equation}\label{eq16}
\begin{aligned}
& \mathbf{w}'= \mathbf{A}{\mathbf{w}} - \mathbf{B}, \ \text{где} \\
& \mathbf{A}=\left[\frac{1}{2}\mathbf{I} + C_{r}{\mathbf{X}}^{\text{T}}{\mathbf{X}}\right]^{\frac{1}{2}},\\
& \mathbf{B}=C_r\left[\frac{1}{2}\mathbf{I} + C_{r}{\mathbf{X}}^{\text{T}}{\mathbf{X}}\right]^{-\frac{1}{2}}
{\mathbf{X}}^{\text{T}}\mathbf{s}.
\end{aligned}
\end{equation}
Тогда, учитывая, что 
\begin{equation*}
C_r\mathbf{s}^{\text{T}}\mathbf{s}-
C_r^2\left\|\left[\frac{1}{2}\mathbf{I} + C_{r}{\mathbf{X}}^{\text{T}}{\mathbf{X}}\right]^{-\frac12}
{\mathbf{X}}^{\text{T}}\mathbf{s}\right\|^2
=\text{const},
\end{equation*}
задача оптимизации принимает вид: 
\begin{equation}\label{eq17}
\begin{aligned}
& \underset{\mathbf{w}', b_i, \xi_{ij}}{\text{minimize:}}
& & \frac{1}{2} \|\mathbf{w}'\|^2 + C\sum\limits_{ij}\xi_{ij} \\
& \text{subject to:}
& & y_{ij}[\left(\mathbf{A}^{-1}\left(\mathbf{w}'+\mathbf{B}\right)\right)^{\text{T}}\mathbf{x}_{ij} - b_i]-1+\xi_{ij} \geq 0, \\
&&& \xi_{ij} \geq 0, \\
&&&i\in\{1,\dots,P\}, \\
&&&j\in\{0,\dots,D\}.
\end{aligned}
\end{equation}
Введем обозначение:
\begin{equation}\label{eq18}
\hat{\mathbf{X}} = (\mathbf{A}^{-1})^{\text{T}}\mathbf{X}.
\end{equation}

Найдём двойственную задачу:
\begin{equation}
\begin{aligned}
& \mathcal{L}(\mathbf{w}', \mathbf{b}, \mathbf{\xi}, \lambda, r) = \frac{1}{2} \|\mathbf{w}'\|^2 + C\sum\limits_{ij}\xi_{ij} - 
\sum_{ij}{\lambda_{ij} \left( y_{ij}[\langle \mathbf{A}^{-1}\left(\mathbf{w}'+\mathbf{B}\right), \mathbf{x}_{ij}\rangle - b_i]-1+\xi_{ij} \right)} -
\sum_{ij}{r_{ij}\xi_{ij}} \\
& \frac{\partial{\mathcal{L}}}{\partial{\mathbf{w}'}} = \mathbf{w}' - \sum{\lambda_{ij}y_{ij}\langle \mathbf{A}^{-1}\left(\mathbf{w}'+\mathbf{B}\right), \mathbf{x}_{ij} \rangle}'_{w'} = 0 \rightarrow \mathbf{w}' = \sum_{ij}{\lambda_{ij}y_{ij}\mathbf{A}^{-1} \mathbf{x}_{ij}} \\
& \forall i, 
\frac{\partial{\mathcal{L}}}{\partial{b_i}} = \sum_j{\lambda_{ij}y_{ij}} = 0 \\
& \forall (i, j), \frac{\partial{\mathcal{L}}}{\partial{\xi_{ij}}} = C - \lambda_{ij} - r_{ij} = 0 \rightarrow \lambda_{ij} + r_{ij} = C
\end{aligned}
\end{equation}
\begin{equation}
\begin{aligned}
& \mathcal{L}(\lambda, r) = \frac{1}{2} \langle \sum_{ij}{\lambda_{ij}y_{ij}\mathbf{A}^{-1} \mathbf{x}_{ij}}, \sum_{ij}{\lambda_{ij}y_{ij}\mathbf{A}^{-1} \mathbf{x}_{ij}} \rangle + C\sum\limits_{ij}\xi_{ij} - 
\sum_{ij}{\lambda_{ij} y_{ij}\langle \mathbf{A}^{-1} \sum_{ij}{\lambda_{ij}y_{ij} \mathbf{A}^{-1}  \mathbf{x}_{ij}}, \mathbf{x}_{ij}\rangle} - \\ 
& - \sum_{ij}{\lambda_{ij} y_{ij}\langle \mathbf{A}^{-1}\mathbf{B}, \mathbf{x}_{ij} \rangle} + \sum_{ij}{\lambda_{ij} y_{ij}b_i} + \sum_{ij}{\lambda_{ij}}  - \sum_{ij}{\lambda_{ij}\xi_{ij}}- \sum_{ij}{r_{ij}\xi_{ij}} = \\
& = \frac{1}{2} \sum_{ij}\sum_{pq} {\lambda_{ij}\lambda_{pq}y_{ij}y_{pq}\langle \mathbf{A}^{-1} \mathbf{x}_{ij}, \mathbf{A}^{-1} \mathbf{x}_{pq} \rangle} + \sum_{ij}{\xi_{ij}(C - \lambda_{ij} - r_{ij})} - \sum_{ij}\sum_{pq} {\lambda_{ij}\lambda_{pq}y_{ij}y_{pq}\langle \mathbf{A}^{-2} \mathbf{x}_{ij},  \mathbf{x}_{pq} \rangle} - \\
& - \sum_{ij}{\lambda_{ij} y_{ij}\langle \mathbf{A}^{-1}\mathbf{B}, \mathbf{x}_{ij} \rangle} + \sum_{i}{b_i \sum_j{\lambda_{ij} y_{ij}}} + \sum_{ij}{\lambda_{ij}}
\end{aligned}
\end{equation}
\\
\begin{enumerate}
	\item $\sum_{ij}{\xi_{ij}(C - \lambda_{ij} - r_{ij})} = 0$ из ограничений;
	\item $\sum_{i}{b_i \sum_j{\lambda_{ij} y_{ij}}} = 0$ из ограничений;
	\item $\langle  \mathbf{A}^{-2} \mathbf{x},  \mathbf{x} \rangle = \langle  \mathbf{A}^{-1} \mathbf{x}, \mathbf{A}^{-1} \mathbf{x} \rangle$, т.к. $\mathbf{x}^{\top}\mathbf{A}^{-2} \mathbf{x} =  \mathbf{x}^{\top}\mathbf{A}^{-1}\mathbf{A}^{-1} \mathbf{x} = (\mathbf{A}^{-1} \mathbf{x})^{\top}\mathbf{A}^{-1} \mathbf{x}$
\end{enumerate}

$\rightarrow$ введём обозначение:
$\widehat{\mathbf{X}} = (\mathbf{A}^{-1})^{\text{T}}\mathbf{X} = \mathbf{A}^{-1}\mathbf{X}$, т.к. $ \mathbf{A}$ симметрична.
\begin{equation}
\begin{aligned}
& \mathcal{L}(\lambda) = - \frac{1}{2} \sum_{ij}\sum_{pq} {\lambda_{ij}\lambda_{pq}y_{ij}y_{pq}\langle \widehat{\mathbf{x}_{ij}}, \widehat{\mathbf{x}_{pq}} \rangle} - 
\sum_{ij}{\lambda_{ij}y_{ij}\langle \mathbf{A}^{-1}\mathbf{B}, \mathbf{x}_{ij} \rangle} + \sum_{ij}{\lambda_{ij}} = \\
& = - \frac{1}{2} \sum_{ij}\sum_{pq} {\lambda_{ij}\lambda_{pq}y_{ij}y_{pq}\langle \widehat{\mathbf{x}_{ij}}, \widehat{\mathbf{x}_{pq}}  \rangle} + 
\sum_{ij}{\lambda_{ij} \left(1  - y_{ij}\langle \mathbf{B}, \widehat{\mathbf{x}_{ij}} \rangle \right)}
\end{aligned}
\end{equation}
Двойственная задача: $\argmax_{\lambda} \mathcal{L}(\lambda)$

Значит, исходная задача по теореме Каруша-Куна-Таккера эквивалентна двойственной:
\begin{equation}\label{eq19}
\begin{aligned}
& \underset{\lambda_{ij}}{\text{minimize:}}
& & \frac{1}{2}\sum\limits_{(i,j),(p,q)}\lambda_{ij}\lambda_{pq}y_{ij}y_{pq}\langle \widehat{\mathbf{x}}_{ij},\widehat{\mathbf{x}}_{pq}\rangle + \sum_{ij}{\lambda_{ij} \left(y_{ij}\langle \mathbf{B}, \widehat{\mathbf{x}_{ij}} \rangle - 1 \right)} \\
& \text{subject to:}
& & 0\leq\lambda_{ij} \leq C, \\
&&& \forall i, \sum_j{\lambda_{ij}y_{ij}} = 0 \\
&&&i\in\{1,\dots,P\}, \\
&&&j\in\{0,\dots,D\}.
\end{aligned}
\end{equation}

\section{Эксперимент}


\section{Заключение}


\begin{thebibliography}{1}
	\bibitem{classification} 
	Maria Kadukova, Sergei Grudinin.
	\textbf{Convex-PL: a novel knowledge-based potential for protein-ligand interactions deduced from structural databases using convex optimization}.
	Journal of Computer-Aided Molecular Design, October 2017, Volume 31, Issue 10, pp 943–958.
	
	\bibitem{D3R}
	Maria Kadukova and Sergei Grudinin.
	\textbf{Docking of small molecules to farnesoid X receptors using AutoDock Vina with the Convex-PL potential : lessons learned from D3R Grand Challenge 2}.
	J. Comput.-Aided Mol. Des., 2017. 
	
	\bibitem{regression} 
	Sergei Grudinin, Maria Kadukova, Andreas Eisenbarth, Simon Marillet, Frédéric Cazals.
	\textbf{Predicting binding poses and affinities for protein-ligand complexes in the 2015 D3R Grand Challenge using a physical model with a statistical parameter estimation}.
	J Comput Aided Mol Des. 2016 Sep;30(9):791-804. Epub 2016 Oct 7.
	
	\bibitem{convex_optimization} 
	S.P. Boyd and L. Vandenberghe.
	\textbf{Convex optimization}.
	Cambridge Univ Press, 2004.
	
%	\bibitem{liblinear} 
%	Rong-En Fan, Kai-Wei Chang, Cho-Jui Hsieh, Xiang-Rui Wang, Chih-Jen Lin.
%	\textbf{LIBLINEAR: A Library for Large Linear Classification}.
%	The Journal of Machine Learning Research archive, Volume 9, 6/1/2008, Pages 1871-1874.

	
\end{thebibliography}

\end{document}
