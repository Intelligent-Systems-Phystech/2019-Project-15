\documentclass[12pt,twoside]{article}
\usepackage[utf8]{inputenc}
\usepackage[russian]{babel}

\title{Рецензия на статью "Поиск символов в художественных изображениях"}
\author{Козинов~А.\,В.; рецензент: Анастасия Грачёва}

\begin{document}
	\maketitle
	В работе рассматривается проблема распознавания художественного изображения, содержащего символы, в зависимости от контекста. Не очень понятно, является ли данная задача актуальной для научного сообщества, но у неё есть потенциал для презентации подхода, интересного самого по себе.
	
	Во введении дано определение такому неочевидному понятию, как символ, а также имеется
	пример работы, в которой решается похожая задача, и указаны отличия авторского подхода.  Но не хватает более полного обзора существующих методов решения задачи со ссылками на соответствующие работы.
	
	Кроме того, по работе имеются следующие замечания:
	\begin{enumerate}
	\item Следует указать конкретный набор данных, на которых будет \\
	производиться обучение и
	тестирование алгоритма.
	\item В описании данных не хватает описания множества 
	рассматриваемых классов - задано ли оно заранее, какого 
	оно размера и т.д.
	\item Не хватает нескольких разделов статьи: теоретического описания алгоритма, описания
	эксперимента, результатов эксперимента и \\
	заключения.
	\item Также в работе присутствуют опечатки, незакрытые скобки.
	
	\end{enumerate}

	Таким образом, статья не может быть опубликована в данный момент, так как требует значительных доработок.
	
\end{document}
